\documentclass[a4paper,12pt]{article}
\usepackage[UTF8]{ctex}
\usepackage{listings}
\lstset{
	breaklines,                                 % 自动将长的代码行换行排版
	extendedchars=false,                        % 解决代码跨页时,章节标题,页眉等汉字不显示的问题
	backgroundcolor=\color[rgb]{0.96,0.96,0.96},% 背景颜色
	keywordstyle=\color{blue}\bfseries,         % 关键字颜色
	identifierstyle=\color{black},              % 普通标识符颜色
	commentstyle=\color[rgb]{0,0.6,0},          % 注释颜色
	stringstyle=\color[rgb]{0.58,0,0.82},       % 字符串颜色
	showstringspaces=false,                     % 不显示字符串内的空格
	numbers=left,                               % 显示行号
	captionpos=t,                               % title在上方(在bottom即为b)
	frame=single,                               % 设置代码框形式
	rulecolor=\color[rgb]{0.8,0.8,0.8},         % 设置代码框颜色
}  



%\usepackage{xeCJK}
\usepackage{times}
\usepackage{setspace}
\usepackage{fancyhdr}
\usepackage{graphicx}
\usepackage{wrapfig}
\usepackage{array}  
\usepackage{fontspec,xunicode,xltxtra}
\usepackage{titlesec}
\usepackage{titletoc}
\usepackage[titletoc]{appendix}
\usepackage[top=30mm,bottom=30mm,left=20mm,right=20mm]{geometry}
\usepackage{cite}
\usepackage{listings}
\usepackage[hidelinks]{hyperref}
\usepackage[framed,numbered,autolinebreaks,useliterate]{mcode} % 插入代码
\XeTeXlinebreaklocale "zh"
\XeTeXlinebreakskip = 0pt plus 1pt minus 0.1pt



%---------------------------------------------------------------------
%	章节标题设置
%---------------------------------------------------------------------
\titleformat{\chapter}{\centering\zihao{-1}\heiti}{实验\chinese{chapter}}{1em}{}
\titlespacing{\chapter}{0pt}{*0}{*6}

%---------------------------------------------------------------------
%	摘要标题设置
%---------------------------------------------------------------------
\renewcommand{\abstractname}{\zihao{-3} 摘\quad 要}

%---------------------------------------------------------------------
%	参考文献设置
%---------------------------------------------------------------------
%\renewcommand{\bibname}{\zihao{2}{\hspace{\fill}参\hspace{0.5em}考\hspace{0.5em}文\hspace{0.5em}献\hspace{\fill}}}

%---------------------------------------------------------------------
%	引用文献设置为上标
%---------------------------------------------------------------------
\makeatletter
\def\@cite#1#2{\textsuperscript{[{#1\if@tempswa , #2\fi}]}}
\makeatother

%---------------------------------------------------------------------
%	目录页设置
%---------------------------------------------------------------------
%\titlecontents{chapter}[0em]{\songti\zihao{-4}}{\thecontentslabel\ }{}
%{\hspace{.5em}\titlerule*[4pt]{$\cdot$}\contentspage}
%\titlecontents{section}[2em]{\vspace{0.1\baselineskip}\songti\zihao{-4}}{\thecontentslabel\ }{}{\hspace{.5em}\titlerule*[4pt]{$\cdot$}\contentspage}
%\titlecontents{subsection}[4em]{\vspace{0.1\baselineskip}\songti\zihao{-4}}{\thecontentslabel\ }{}{\hspace{.5em}\titlerule*[4pt]{$\cdot$}\contentspage}


\usepackage{ctex}
\usepackage{listings}
\usepackage{xcolor}
\usepackage{graphicx}
\usepackage{subfig}
\usepackage{array}
\begin{document}
	\section{第一问}
	\subsection{累计规模法介绍}
	记总量为$N$,各单元的规模为$\{M_i\}_{i=1,\cdots,N}$,累计规模$C_k=\sum_{i=1}^{k}M_i,k=1,\cdots,N$,其中$C_0=0$。累计规模法会生成
	区间列$\{I_i\}_{i=1,\cdots,N}$,其中$I_i=[C_{i-1}+1,C_{i}]$。

	然后在$[1,max\{M_1,\cdots,M_N\}]$生成$n$个随机整数$\{Z_i\}_{i=1,\cdots,n}$,其中$n$为样本容量。
	对每一个$i=1,\cdots,n$,若$Z_i$落在了区间$I_j$中,则第j个总体入样。
	\subsection{R语言PPS抽样}

    以数据集中的popn(人口数)作为规模,确定每个州被抽中的概率。设第i个州的人口数
    为$P_i$,则其被抽中的概率应为$\phi_i=\frac{P_i}{\sum_{j=1}^{n}P_j}$。R语言代码如下:

	\begin{lstlisting}[language=r,breaklines]
		df<-read.csv("statepps.csv")
		library("sampling")
		library("survey")
		N=nrow(df)
		n=10
		pik=inclusionprobabilities(df$popn, n)
		s=UPmultinomial(pik)
		df.pps = df[s!=0, ]
		Z=pik[s!=0]/n #每个单元被抽中的概率
		Q=s[s!=0] #每个单元被抽中的次数
	\end{lstlisting}
	
	\subsection{参数估计}
	
	总体总量的无偏估计应为:
	$$\hat{Y}_{HH}=\frac{1}{n}\sum_{i=1}^{n}\frac{y_i}{\phi_i}$$

	方差的无偏估计应为:
	$$Var(\hat{Y}_{HH})=\frac{1}{n(n-1)}\sum_{i=1}^{n}(\frac{y_i}{\phi_i}-\hat{Y}_{HH})^2$$

	估计标准误为:
	$$SE=\sqrt{Var(\hat{Y}_{HH})}$$

	R语言代码如下:
	\begin{lstlisting}[language=r,breaklines]
		Yhh=mean(df.pps$counties/Z*Q)
		vars=1/(n*(n-1))*sum((df.pps$counties/Z-Yhh)^2*Q)
		std=sqrt(vars)
	\end{lstlisting}
	输出结果为:$\hat{Y}_{HH}=3686.316$,$SE=655.797$
	\section{第二问}
	\subsection{参数估计原理}

	先用ssu样本估计各psu的$\hat{Y}_{HH_i}$,各psu的总量的估计为:
	$$\hat{Y}_{HH_i}=\frac{M_i}{k_i}\sum_{j=1}^{k_i}y_{ij}$$
	其中$k_i$是第i个psu的中抽取的ssu的样本个数。
	
	然后可以求总体总量的估计:$$\hat{Y}_{HH}=\frac{1}{n}\sum_{i=1}^{n}\frac{\hat{Y}_{HH_i}}{\phi_i}$$

	方差的无偏估计应为:
	$$Var(\hat{Y}_{HH})=\frac{1}{n(n-1)}\sum_{i=1}^{n}(\frac{\hat{Y}_{HH_i}}{\phi_i}-\hat{Y}_{HH})^2$$

	估计标准误为:
	$$SE=\sqrt{Var(\hat{Y}_{HH})}$$

	\subsection{计算}

	本题中,psu样本数$n=10$。

	用题中数据计算的结果如下表。

	\newpage

	\begin{table}[]
		\centering
		\begin{tabular}{lll}
		学院编号 & $\phi_i$ & $\hat{Y}_{HH_i}$ \\
		14   &     0.08054523         &          113.75                           \\
		23   &     0.03097893         &          31.25                           \\
		9    &     0.05947955         &          12                           \\
		14   &     0.08054523         &          48.75                           \\
		16   &     0.002478315        &          2                           \\
		6    &     0.07682776         &          139.5                           \\
		14   &     0.08054523         &          65                           \\
		19   &     0.07682776         &          77.5                           \\
		21   &     0.07558860         &          122                           \\
		11   &     0.05080545         &          225.5                          
		\end{tabular}
		\end{table}

	算得其总量的估计为$$\hat{Y}_{HH}=\frac{1}{n}\sum_{i=1}^{n}\frac{\hat{Y}_{HH_i}}{\phi_i}\approx 1371.9$$

	标准误为:$$SE=\sqrt{\frac{1}{n(n-1)}\sum_{i=1}^{n}(\frac{\hat{Y}_{HH_i}}{\phi_i}-\hat{Y}_{HH})^2} \approx 372.980$$
	\subsection{Lahiri方法介绍}
	设总体中有N个个体,第i个单元的规模为$M_i$,涉及的抽样均为有放回的抽样。

	1. 抽取随机数$i\in \{1,\cdots,N\}$

	2. 抽取随机数$M\in \{1,\cdots,max\{M_1,\cdots,M_N\}\}$

	3. 若$M_i\geq M$,则第i个样本入样,否则不入样。

	4. 重复上述过程,直至抽满n个单元
\end{document}=====
